Todo list for changing the docs directory:
------------------------------------------

1. Ssh to my server.
2. create a working-directory (for example: mkdir working_directory) in your homedir
3. go to the working-directory (cd working-directory)
4. type "cvs checkout docs" to retrieve the directory with all docs concerning the mud
5. go to the docs directory (cd docs)
6. look at and change anything you like (using vi)
7. go to working-directory (cd ..)
8. type "cvs commit docs" to commit any and all changes you made to the documents (it will ask for some comments concerning the changes you made)
9. type "cvs release -d docs" to release the docs directory and remove it from the "working-directory"

File Permissions
----------------
File permissions necessary for a good use of cvs are:
- rwx on directories
- r on files


Building the repository:
------------------------
export CVSROOT="/karchan2/mud/cvsroot/"
create directory "cvsroot" in /karchan2/mud/
alias cvs="/karchan2/cvs/bin/cvs"

Checking out files:
/* the -P is necessary in order not to retrieve empty directories */
cvs checkout -P <directoryname>

Adding files to the directories:
cvs add <filename>
cvs commit <filename>

Removing files from directory:
cvs remove <filename>
cvs commit <filename>

Checking logs and history of certain files:
cvs log <filename>
cvs history <filename>

Saving an entire directory and releasing it and removing it:
/* in order to continue with business */
cvs commit <directory>
cvs release -d <directory>

Accessing CVS files remotely:
-----------------------------

First login:
cvs -d :pserver:karn@linux706.dn.net:/karchan2/mud/cvsroot login         

Then use all the old standard cvs commands as explained above:
cvs -d :pserver:karn@linux706.dn.net:/karchan2/mud/cvsroot checkout docs

Then use the following command to log out again:
cvs -d :pserver:karn@linux706.dn.net:/karchan2/mud/cvsroot logout         
(very important!!! otherwise some files with your password for your host might be mistakenly left behind)

